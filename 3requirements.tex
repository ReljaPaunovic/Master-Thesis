\chapter{Requirements}

To design a solution, environment around the problem needs to be understood, specifically, stakeholders involved and what needs they have. This chapter explains several industrial use cases with aim to familiarize the reader with environment Contracting Service is tailored for, followed by concrete requirements of this solution. Since there are countless number of different use cases, not everything can be covered, especially because new applications are continuously emerging. Therefore, this solution needs to be flexible to cover most of the existing and future use cases.  

\section{Context}

This section describes three representative industrial use cases in order to help reader understand the range of applications. All use cases have a common core even though they seem very different from each other as it can be seen from the following text.

\subsection{Smart Traffic}

Recently, smart traffic is emerging as promising trend with the idea to automate transportation of goods and people. Big corporations, such as Google, Tesla, Mercedes and Uber have already joined the race with their models but many technological, legal and business obstacles are holding back its deployment. Integration of these vehicles into regular, non-automated, traffic is a big challenge because it needs to take into account human error and correct it if possible.

One interesting example are smart convoys, driver-less trucks transporting goods in a convoy. For this to be secure, communication between trucks needs to be in real-time so that all, for example, obstructions noticed by sensors on truck in front are conveyed to the rest in time for them to react (slowing down or avoiding obstruction). It is also very important that only authorized people have access to data convoy produces, because if something gets tampered with, human lives are at stake along with structural damages. 

For these security reasons not all stakeholders should have access to all the data but only to what is necessary for their operations. Stakeholders involved are: manufacturers; maintenance company; users; and third parties. Firstly, manufacturers should not have access to data that discloses anything that is confidential for truck users, for example, exact location of the convoy in any time because this information can be used to get insight into operations of users. Secondly, manufacturers might not want to give out all information to their users, either because of confidentiality or because it might discredit them. Thirdly, maintenance companies should have only information about the state of the parts of truck. How many times have the breaks been used, level of motor oil, gas and similar which they need in order to schedule maintenance control. Finally, third parties could be any company, or public, that benefits from data these trucks produce and fit into business models of manufacturers or users. For example, traffic control application that provides information about how many convoys are on a particular road so that if the number is too high, traffic jams are expected.

\subsection{Connected Goods}

The  servitisation  of  physical  goods  will  be  of  strategic  importance  for the  manufacturing  industry, where instead of selling parts and machines it will be possible to sell engine hours, kilometers and similar. The vision here is that goods will remember how they were made and produce data throughout whole cycle of their usage, even giving insights into customer satisfaction with those goods. In this case, privacy is a big obstacle, because it is hard to assure customer that data you collected in their home or workspace is not going to be used by anybody they do not want to.

Consider a scenario where all goods in your apartment from carton of milk to air-conditioning are equipped with sensors. Milk carton might posses a heat sensor which alerts when milk is being kept in a warm place for too long, labeling it as spoiled and ordering fresh one from a local marketplace. Same data can be used by for statistical purposes by a third party, to determine, for example, how much milk is being wasted by a nation and use it to adjust size of milk cartons or predict peaks in milk consumption. With some more complex goods, such as air-conditioning, predictive maintenance can be realized with sensors that tell if a particular part is wearing off and alert maintenance service specified by user or manufacturer of that machine. Subset of the data generated by air-conditioning can be used by health organizations to determine whether people are living in unhealthy environments, for example, by checking the ration between how many times air filters have been changed over hours of usage.

Scenarios described in previous paragraph are just few out of many possible ones and it is impossible to tell which ones will be implemented. Although, we have to prepare for the future by designing a flexible system that can withstand rapid changes in the ecosystem.

\subsection{Smart Crane}

KoneCranes have donated a smart crane to Aalto Industrial Campus as a tool for research. Following information was gathered through interviews with researchers and with KoneCranes and it will be used as a representative example for the rest of this work.

This crane has many sensors attached to make it automated and smart as possible. These sensors
bring many features, namely {\bf load weighting} (which can also be used to detect
whether the crane is stuck somewhere), {\bf remote monitoring } of the position of the crane,
{\bf signaling} when some error or warning has occurred, {\bf live video feed } from camera attached
above the hook (at the moment, not used for automation but could be supported in the future,
for example, image processing to detect obstructions) and more.

There are multiple stakeholders in this ecosystem and all of them require some of the data
that Crane generates, although not all information should be available to them, only
the bare minimum that is required by their business. These stakeholders are KoneCranes, 
maintenance service (part of the KoneCranes) and users of the crane. Currently, KoneCranes has access to all the data Crane generates, but in order to 
increase security, this should not be the case.

\subsubsection{Maintenance}

Maintenance service makes use of usage data (alarms, usage parameters) to monitor and predict maintenance needs of the cranes. They also use the data together with the customer, for example, to review their maintenance spend of the assets, study patterns to reveal relationships between variables and more.

For example, part of the crane that needs to be changed most often are the breaks.
Breaks have limited number of uses and this number is approximately known
(there is a regulation in place that requires brakes to be changed regularly).
Information like, how many times has the Crane been moved from idle state,
should be available to maintenance service so they know when do they need to
change the breaks (or some other part of the Crane). Crane smart sensors can also 
detect some irregularities and issue a warning (or error),
which is useful information for the maintenance companies.

Information like this should be disclosed to the maintenance service, while
restricting the access to, for example, position of the crane in a certain
point in time, or any kind of data that can be used to infer the processes
inside the factory.

\subsubsection{Users of the Crane}

Inside a factory there are different actors with different privileges. Worker that
operates Crane does not need to know the history of the movements of the Crane, or 
condition of the brakes. Thus, he should only have access to information he needs, e.g. 
from where to where the load needs to be moved.
Manager of those workers on the other hand should be able to monitor all the Cranes in
the factory.

In the future, machine to machine communication will be utilized much more, for example, 
in a setting with two Cranes operating (automated, not by human) in the same room they 
should be able to signal to each other with their planned path in order to avoid collisions
and to optimize the process. In this way, need for a centralized control is eliminated (or at
least minimized).

\subsubsection{KoneCranes}

KoneCranes needs access to some of the data that Crane generates, in order to make product improvements, react to possible reliability problems and get better specification for new product generations, adjust warranties, and such.
They analyze all types of data that Crane generates: manufacturing data (component lists, manufacturing dates, and more), usage (starts, lifting hours, loads, and more) and sensor data (vibrations, temperature, and more), and maintenance data (maintenance task history, ordered materials, and more).

At the moment KoneCranes has access to all the data Crane generates and their customers are aware
of that. For privacy reasons, and in environment with multiple manufacturers, data access should be restricted to only what is needed for a specific role. 

\section{Security}

\subsection{Authentication}

\fixme{Dynamic policies should be mentioned - talk about role based access control}

\subsection{Authorization?}

\section{User Experience}

\fixme{Managing devices for a machine, for a role at same time, Niemen 10?}
