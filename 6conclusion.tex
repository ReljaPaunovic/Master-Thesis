\chapter{Conclusion}
\label{chapter:conclusion}

This work investigates use cases of smart industrial machines, mostly focusing on the smart crane donated to Aalto Industrial campus, in order to create a contracting service for management of access to smart machines. The research aim of this thesis, defined in section \ref{section:researchQuestion}, is the solution for easy and secure access management for large number of devices in industrial setting. Three sub questions of the research describe the phases necessary in order to answer the main question. These fazes are: gathering and listing requirements of the solution, forming necessary components of the solution and evaluating the solution with respect to requirements. 

Main findings of this work are presented in the chapter \ref{chapter:implementation} showing the necessary components of this solution. Four types of accounts were deemed necessary, Administrator account, Manufacturer account, Customer account, and User account. Administrator has a role to create and manage Manufacturer accounts, since the creation of accounts is not possible for anyone, only for verified Manufacturers. Manufacturer account needs to create Customer accounts on request of customers, add machines and devices to the system and assign them to the customers. Customer account can then either, create User accounts and assign machines or devices to them, or can manipulate access to the machines or devices by inserting or removing policies from the Policy Database. User account has a similar role as a Customer account, only he can not create new accounts or assign devices to other accounts. 

This solution for management of machines and devices have a small scalability issue in terms of user experience. When a large number of devices are added to the system, the overview of machines and devices becomes cluttered. Thus, finding a right machine or device can be problematic. This issue can be solved by including a search option. Search option should take into account number of devices, types of devices or machines, and names of machines or devices. Moreover in terms of performance, increasing number of devices pose some scalability issues when contacting the Policy Database. These issues can be simply solved by distributing the Policy Database on several locations. However, smart ways of distributing the Policy Database is out of the scope of this work.

Implementation described in chapter \ref{chapter:implementation} is constructed according to the requirements described in section \ref{chapter:requirements}. Requirements were extracted from context and grouped into three main categories general requirements, manufacturer requirements and customer requirements. General requirements are further divided into security, user experience and functional requirements related to all roles in the system. These requirements are then evaluated in section \ref{chapter:evaluation}.

Chapter \ref{chapter:background} gives background information about the problem my solution is solving. Firstly, it introduces the Internet of Things (IoT) and then compares it with Industrial Internet of Things (IIoT) in order to highlight the differences in requirements between them. Secondly, it briefly describes issues regarding standardization, security and privacy. Furthermore, current existing solutions are described and their shortcomings highlighted. Finally, most notable standards are described in more detail (LWM2M and W3C WOT), followed by a work by Kantola et. al. \cite{Kantola,5480987} on which this solution relies to provide a desired functionality. 