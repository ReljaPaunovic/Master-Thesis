\chapter{KoneCranes Use Case}
\label{chapter:first-appendix}

KoneCranes have donated a smart Crane to Aalto Industrial Campus. This Crane has 
many sensors attached to make it automated and smart as possible. These sensors
bring many features, namely {\bf load weighting} (which can also be used to detect
whether the crane is stuck somewhere), {\bf remote monitoring } of the position of the crane,
{\bf signaling} when some error or warning has occurred, {\bf live video feed } from camera attached
above the hook (at the moment, not used for automation but could be supported in the future,
for example, image processing to detect obstructions) and more.

There are multiple actors in this ecosystem and all of them require some of the data
that Crane generates, although not all information should be available to them, only
the bare minimum that is required by their business. These actors are KoneCranes, 
maintenance service (part of the KoneCranes) and users of the crane. Currently, KoneCranes has access to all the data Crane generates, but in order to 
increase security, this should not be the case.

\section*{Maintenance}

Maintenance service makes use of usage data (alarms, usage parameters) to monitor and predict maintenance needs of the cranes. They also use the data together with the customer, for example, to review their maintenance spend of the assets, study patterns to reveal relationships between variables, etc.

For example, part of the crane that needs to be changed most often are the breaks.
Breaks have limited number of uses and this number is approximately known
(there is a regulation in place that requires brakes to be changed regularly).
Information like, how many times has the Crane been moved from idle state,
should be available to maintenance service so they know when do they need to
change the breaks (or some other part of the Crane). Crane smart sensors can also 
detect some irregularities and issue a warning (or error),
which is useful information for the maintenance companies.

Information like this should be disclosed to the maintenance service, while
restricting the access to, for example, position of the crane in a certain
point in time, or any kind of data that can be used to infer the processes
inside the factory.

\section*{Users of the Crane}

Inside a factory there are different actors with different privileges. Worker that
operates Crane does not need to know the history of the movements of the Crane, or 
condition of the brakes. Thus, he should only have access to information he needs, e.g. 
from where to where the load should be moved.
Manager of those workers on the other hand should be able to monitor all the Cranes in
the factory.

In the future, machine to machine communication will be utilised much more, for example, 
in a setting with two Cranes operating (automated, not by human) in the same room they 
should be able to signal to each other with their planned path in order to avoid collisions
and to optimize the process. In this way, need for a centralized control is eliminated (or at
least minimized).

\section*{KoneCranes}

KoneCranes also wants to access some of the data that Crane generates, in order to make product improvements, react to possible reliability problems and get better specification for new product generations, adjust warranties, etc..
They analyze all types of data that Crane generates: manufacturing data (component lists, manufacturing dates, etc.), usage (starts, lifting hours, loads, etc.) and sensor data (vibrations, temperature, etc.), and maintenance data (maintenance task history, ordered materials, etc.).

At the moment KoneCranes has access to all the data Crane generates and their customers are aware
of that. For privacy reasons, and in environment with multiple manufacturers, data access should be restricted to only what is needed for a specific role. 
